\documentclass[a4paper, 11pt, twoside]{article}
\usepackage[margin = 2cm]{geometry}

\usepackage[utf8]{inputenc}
\usepackage[T1]{fontenc}
\usepackage[francais]{babel}
\usepackage{hyperref}

\title{\texttt{picOS} : Petite Implémentation en C d'un OS}
\author{Martin Janin, Antonin Reitz, Jules Saget}
\date{2018}

\begin{document}

\maketitle

\section{Vue d'ensemble}

\subsection{Objectifs et spécifications}

\texttt{picOS} est un micronoyau pour \texttt{i386-elf}. Dans un premier temps, nous
voulions coder un système d'exploitation pour architecture \texttt{x86\_64}, avec
pagination et système de fichier. Au final, nous avons un début d'ordonnanceur, une 
interface utilisateur - machine basique et un début de pagination.

Toutes les fonctionnalités de notre OS ont été intégralement codées par nos
soins, à l'exception notable du chargeur d'amorçage. Nous nous sommes beaucoup
inspirés des articles d'OSDev (\url{osdev.org}), mais l'ensemble de notre code est
original.

\subsection{Architecture générale}

L'architecture est inspirée du \emph{Meaty Skeleton} d'OSDev. Elle se divise en
trois grandes parties~:
\begin{itemize}
  \item Le noyau : le dossier \texttt{kernel} contient toutes les fonctions
    relatives au noyau. Cela comporte notamment tout ce qui est dépendant du
    processeur (éditeur de lien, périphérique clavier, $\ldots$) ainsi que toutes
    les sources permettant de compiler le noyau.
  \item La bibliothèque C : le dossier \texttt{libc} contient toutes les
    fonctions relatives au noyau. Cela comporte l'interface pour les appels
    système, l'ordonanceur, les structures de données et l'allocation mémoire.
  \item Les scripts : le dossier \texttt{scripts} contient tous les scripts
    nécessaires à la compilation. C'est ce qui relie les différentes parties
    entre elles pour construire l'image disque.
\end{itemize}

\section{Aspects techniques}

\subsection{Utilitaires externes}

Pour nous faciliter la tâche, nous avons eu recours à différents utilitaires
externes. Outre les outils standards (comme l'utilisation de Git ou de
Makefiles), nous nous sommes servis du chargeur d'amorçage GRUB et du simulateur
de machine virtuelle Qemu. A noter que nous nous sommes également servis de GDB pour inspecter la mémoire et les registres pendant l'exécution ; ainsi que de objdump, od et readelf pour examiner les exécutables binaires produits.

\subsubsection{Git}

Tout notre code est accessible sur le dépôt Git situé à l'adresse suivante :
\url{https://github.com/Involture/picOS}. La branche Master contient tout le
code compilable (à quelques erreurs près). La branche Testing contient le code
codé mais pas stable, ou pour lequel il manque des fonctions non codées. Enfin,
la branche Rendu contient la version du projet présentée à la soutenance.

\subsubsection{GCC et compilation croisée}

Notre système d'exploitation étant codé en C, nous avons utilisé GCC pour le
compiler. Néanmoins, puisque nous compilons pour une autre machine que celles
sur lesquelles nous travaillons (notre système d'exploitation est conçu pour
architecture \texttt{i386-elf}), nous avons dû reconfigurer GCC pour qu'il fasse de la
compilation croisée.

Les fichiers binaires de cette compilation de GCC sont en annexe du projet, et
ils sont en théorie censés convenir pour les machines tournant sous Unix. Si
cela ne fonctionne pas, il vous suffit de vous-même recompiler GCC pour
\texttt{i386-elf} (tutoriel du cross-compiler \texttt{x86} sur OSDev).

\subsubsection{Makefile}

Le Makefile est celui tiré de l'organisation de projet Meaty Skeleton d'OSDev,
largement modifié.
Les scripts bash ont également été largement améliorés, afin de faciliter la 
compilation, notamment avec l'option --gdb.

\subsubsection{GRUB}

Nous avons étudié la spécification du standard multiboot (1) afin que GRUB puisse
amorcer notre kernel. Le chargement en mémoire du noyau s'avère complexe : dans la
 version avec le paging, le noyau est décomposé en deux parties. En effet, les liens
 de la partie inférieure sont édités pour fonctionner lorsqu'elle est chargée en bas 
 la mémoire, tandis que les liens de la partie supérieure sont édités pour fonctionner
 lorsque qu'elle est chargée en haut de la mémoire (0xFF000000).

\subsection{Ordonnancement}

Pour l'ordonancement des processus, nous avons théorisé et commencé à
implémenter une interface d'ordonnancement non préemptive. L'objectif des appels
systèmes implémentés est de permettre l'utilisation d'autres ordonnanceurs
préemptifs, du côté utilisateur. Ainsi, ces ordonnanceurs pourraient être
rajoutés à la librairie C.
Une interface spécifique est également prévue avec les pilotes de périphériques
pour permettre par exemple au processus de configurer des comptes à rebours.

\subsection{Gestion de la mémoire}

\subsubsection{Pagination}

À l'heure actuelle, la pagination est presque fonctionnelle. Les structures
nécessaires à son implémentation sont présentes, et il ne manque plus que les
fonctions pour manipuler les pages.

Dans notre version de la pagination, nous nous sommes placés dans le cadre
suivant\~:
\begin{itemize}
  \item Les processus ont un tas et une pile de taille fixée (à 1000 pages soit
    4 kio), considérée comme sufisamment grande pour qu'ils n'aient pas besoin
    de demander plus de place.
  \item Les processus sont placés les uns à la suite des autres dans la mémoire.
\end{itemize}

\subsubsection{Allocation mémoire}

L'allocation mémoire est une adaptation de la version la plus performante
présentée dans le cours \emph{Langages de programmation et compilation} de
Jean-Christophe Filliâtre. Elle se base sur la découpe de la mémoire en blocs de
4 octets et sur une utilisation intelligente de l'espace contenu dans les
blocs libres.

\subsection{Interface utilisateur-machine}

\subsubsection{Clavier}

Un driver clavier PS2 est implémenté, le scan code set 2 est utilisé. La keymap utilisée
est la keymap classique AZERTY. Afin de différencier les différentes touches, et les
différents signaux que chacune envoie (pression / relâchement), une disjonction de cas
extrêmement lourde est nécessaire. Celle-ci constitue le seul cas de code généré de notre
projet.
Dans le dossier \texttt{aux\_kbd\_kmp} se trouvent le code Python à l'origine de ce code généré,
produit en fonction de la keymap stockée dans le fichier texte dans le même répertoire.
Un fichier C est ensuite produit (\texttt{gc.c}), qui n'a plus qu'à être sommairement copiée.
(Les modifications relativement rares de cette partie sont à l'origine de la relative lourdeur
de la modification de celle-ci.)
L'implémentation du driver PS2 est classique, il s'agit d'une simple file représetée par un 
tableau circulaire.

\subsubsection{Shell}

Afin de permettre une interaction avec la machine qui commence tout juste à être intéressante, 
un shell avec des commandes basiques a été implémenté. L'essentiel de ces commandes est en lien 
avec le système de fichiers présenté dans la sous-sous-section.
Seules des commandes simples sont actuellement implémentées, bien qu'un système de redirections 
soit désormais simple à implémenter, ne nécessitant que quelques heures de travail.

\subsubsection{Système de fichiers virtuel}

Le malloc optimisé n'a pas été utilisé, notamment car celui-ci avait été réalisé en supposant
que le paging marcherait. Des tableaux de grande taille ont été utilisés pour représenter
la RAM, associés à un malloc simplissime.

Le système de fichiers, qualifié de virtuel car ne correspondant qu'à des variables stockées
dans la RAM, est lui aussi très simple.
Les opérations possibles sont les opérations classiques : le modèle de système de fichiers
réalisé semble permettre l'implémentation relativement simple de commandes plus élaborées ayant
un effet sur celui-ci en quelques heures de travail.

\section{Perspectives d'amélioration}

\subsection{Ce qu'il faudrait coder à court terme}

Actuellement, l'interface de l'ordanceur n'est pas complétement implémentée et
est donc non-fonctionnelle. Quelques heures de travail nous manquent pour avoir
une gestion complète de la mémoire et des processus.

Également, un bug non résolu demeure dans le chargement en mémoire de la section
\texttt{.data} du noyau par GRUB. Ce bug doit être résolu pour permettre de
tester l'ordonanceur et la mémoire.

Quelques commandes shell de plus sont à implémenter en raison de la simplicité de cette
en comparaison de toutes les autres choses à faire.

\subsection{Les fonctionnalités à rajouter}

Nous aurions aimé avoir d'un système de fichier non-virtuel. Nous voulions coder
nos pilotes de disque dur puis implémenter ext2 et éventuellement ext3.

La bibliothèque C (\texttt{libc}) a été à peine entamée. De nombreuses
structures et fonctions restent à coder pour rendre le noyau pleinement
fonctionnel et espérer lancer un processus utilisateur.

Lorsque l'espace utilisateur sera disponible, il faudra également porter le
shell et éventuellement les pilotes du côté utilisateur.

Une fois l'ordonnanceur pleinement implémenté, les pipes pourraient assez 
rapidement suivre.

\subsection{Les erreurs qu'on ne commettra plus}

Dans plusieurs cas, nous nous sommes parfois directement lancés dans des tâches complexes
au lieu de commencer par une version relativement simple et d'augmenter graduellement la difficulté.

De plus, nous avons sous-estimé ce qui a pour nous été le nerf de la guerre, notamment
sur la fin du projet, à savoir la mémoire.

\end{document}
